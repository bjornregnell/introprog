%!TEX encoding = UTF-8 Unicode
%!TEX root = ../compendium.tex

\Assignment{bank}

\begin{Goals}
\item Kunna implementera ett helt program efter given specifikation
\item Förstå hur Java-klasser kan användas i Scala
\item Förstå och bedöma när immutable/mutable såväl som var/val bör användas i större sammanhang
\item Kunna använda sig av kompanjons-objekt
\item Kunna läsa och skriva till fil
\item Kunna söka i olika datastrukturer på olika sätt
\end{Goals}

\subsection{Bakgrund}

I denna uppgiften ska du skriva ett program som håller reda på bankkonton och kunder i en bank. Programmet ska även hålla reda på bankens nuvarande tillstånd, såväl som föregående.
Tillstånden ska vid varje tillståndsförändringar skrivas till fil så att utifall banken skulle krascha finns alla transaktioner som genomförts sparade så att banken kan återställas.

Programmet ska vara helt textbaserat, man ska alltså interagera med programmet via konsollen där en meny skrivs ut och input görs via tangentbordet.

Du ska skriva hela programmet själv, men det ska dock följa de specifikationer som ges i uppgiften, såväl som de objektorienterade principer du lärt dig i kursen.

\subsection{Krav}

Kraven för bankapplikationen återfinns här nedan. För att bli godkänd på denna uppgift måste samtliga krav uppfyllas:

\begin{itemize}
\item Programmet ska ha följande funktioner tillgängliga via menyn:

\begin{itemize}
\item Hitta konton för en viss kontoinnehavare.
\item Sök kontoinnehavare på (del av) namn.
\item Sätta in pengar på ett konto.
\item Ta ut pengar på ett konto.
\item Överföra pengar mellan två olika konton.
\item Skapa ett nytt konto.
\item Ta bort ett befintligt konto.
\item Skriv ut bankens alla konton.
\item Återställa banken till ett tidigare tillstånd vid ett givet datum.
\item Avsluta.
\end{itemize}

\item Programmet ska skapa nytt tillstånd med tidsstämpel och spara gamla varje gång då:
\begin{itemize}
\item Pengar sätts in eller tas ut ifrån ett konto.
\item Pengar överförs mellan två konton.
\item Ett konto skapas.
\item Ett konto tas bort.
\end{itemize}
\item Då ett tillstånd förändras ska detta skrivas till fil.
\item Programdesignen ska följa de specifikationer som är angivna nedan.
\item Inga utskrifter eller inläsningar får göras i klasserna Customer, BankAccount, Bank, State eller Transaction. Allt som berör användargränssnittet ska ske i BankApplication. Det är tillåtet att använda valfritt antalat hjälpmetoder och hjälpklasser i klassen BankApplication.
\item Alla metoder och attribut ska ha lämpliga åtkomsträttigheter.
\item Valet av val/var och immutable/mutable måste vara lämpliga.
\item Din indata måste ge samma resultat som i exemplen (som kommer komma i framtiden) i bilagan.
\item Rimlig felhantering ska finnas, det är alltså önskvärt att programmet inte kraschar då man matar in felaktig input, utan istället säger till användaren att input är ogiltlig.

\end{itemize}

\subsection{Design}
Nedan följer specifikationerna för de olika klasserna bankapplikationen måste innehålla:

BankAccount

\begin{Code}
/**
 * Skaper ett nytt bankkonto åt innehavaren Customer.
 * Kontot tilldelas ett unikt kontonummer och innehåller 
 * inledningsvis 0 kr.
 */
class BankAccount(val holder: Customer);

  /**
   * Sätter in mängden amount pengar på detta konto.
   */
  def deposit(amount: Int): Unit;

  /**
   * Hämtar saldot på konton.
   */
  def getBalance: Int;

/**
   * Tar ut mängden amount pengar från kontot sålänge
   * amount ej överstiger nuvarande saldo.
   * Returnerar true om transactionen lyckades, annars false. 
   */
  def withdraw(amount: Int): Boolean;

\end{Code}

Bank

\begin{Code}
/**
 * Skapar en ny bank utan konton och tillstånd. 
 */
class Bank();

/**
   * Skapar ett nytt konto på banken. Kontonumret som 
   * genereras åt kontot returneras.
   */
  def addAccount(name: String, id: Int): Int

/**
   * Returnerar kontoinnehavaren kopplat till det givna 
   * id-numret, om ingen sådan kund finns returneras 
   * null istället.
   */
  def findHolder(id: Int): Customer;

/**
   * Ta bort kontot med kontonumret accountNbr och returnerar
   * true. 
   * Om inget sådant konto existerar returneras false istället. 
   */
  def removeAccount(accountNbr: Int): Boolean;

/**
   * Returnerar en lista med samtliga bankkontot i banken.
   * Finns inga kontot returneras en tom lista.
   */
  def getAllAccounts(): ArrayBuffer[BankAccount];

/**
   * Returnerar det bankkonto som har kontonummer accountNbr.
   * Finns inget sådant konto returneras null istället.
   */
  def findByNumber(accountNbr: Int): BankAccount;

/**
   * Söker upp alla konton med en innehavaren vars namn 
   * innehåller strängen namePattern och returnerar dessa
   *  i en lista.
   */
  def findByName(namePattern: String): ArrayBuffer[Customer];

/**
   * Genomför en transaktioner i banken och returnerar en sträng 
   * med nödvändig information till konsolen.
   */
  def makeTransaction(transaction: Transaction): String;

/**
   * Återställer banken till det tillstånd som gällde vid
   * tiden date.
   */
  def returnToState(returnDate: Date): String;


\end{Code}

Transaction

\begin{Code}
/**
 * Beskriver en transaktion som används av banken.
 */
abstract class Transaction;

\end{Code}


Customer

\begin{Code}
/**
 * Beskriver en kund med namnet name och personnummer id.
 */
class Customer(val name: String, val id: Int);

\end{Code}



\subsection{Obligatoriska uppgifter}

\Task Implementera klassen Customer

\Task Implementera klassen BankAccount

\Task Implementera de klasser som ska förlänga Transaction

\Task Skapa klassen BankApplication.

\Subtask Klassen BankApplication ska innehålla main-metoden. Det kan vara bra att innan man fortsätter se till att denna klass skriver ut menyn korrekt och kan ta input från tangentbordet som motsvarar de menyval som finns.

\Task Implementera menyval 6 och 8. Testa noga.

\Task Implementera tillstånds funktionaliteten. Varje ny transaktion ska ge upphov till nytt tillstånd.

\Task Implementera klassen Bank utefter de menyval du implementerar.

\Task Implementera menyval 9. Testa noga.


\begin{Code}
/** A simple wrapper of Java.time */
case class Date;
\end{Code}

\subsection{Frivilliga extrauppgifter}

\Task Ej bestämt ännu.

