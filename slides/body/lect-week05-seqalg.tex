%!TEX encoding = UTF-8 Unicode
%!TEX root = ../lect-week05.tex

%%%

\begin{Slide}{Vad är en sekvensalgoritm?}
\begin{itemize} 
\item En algoritm är en stegvis beskrivning av hur man löser ett problem. 
\item En sekvensalgoritm är en algoritm där dataelement i sekvens utgör en viktig del av problembeskrivningen och/eller lösningen.   

\item Exempel: sortera en sekvens av personer efter deras ålder.

\item Två olika principer:
\begin{itemize} 
\item Skapa \Emph{ny sekvens} utan att förändra indatasekvensen
\item Ändra \Emph{på plats} \Eng{in place} i den \Alert{föränderliga} indatasekvensen
\end{itemize}
\end{itemize}

\end{Slide}


\begin{Slide}{Några indexerbara samlingar}
\begin{itemize}
\item Oföränderliga:  
  \begin{itemize} 
  \item Kan \Alert{ej} ändra elementreferenserna: \\ 
    Scala: \Emph{Vector}, \Emph{List}
  \end{itemize}

\item Föränderliga: kan \Alert{ändra} elemententreferenserna
  \begin{itemize} 
  \item Kan \Alert{ej ändra storlek} efter allokering: \\ Scala+Java: \Emph{Array} 
  \item Kan ändra storlek efter allokering: \\ Scala: \Emph{ArrayBuffer} \\ Java: \Emph{ArrayList}
  \end{itemize}
\end{itemize}
\end{Slide}

\begin{Slide}{Algoritm: SEQ-COPY}
\begin{algorithm}[H]
 \SetKwInOut{Input}{Indata}\SetKwInOut{Output}{Resultat}
 \Input{Heltalsarray $xs$} 
 \Output{En ny heltalsarray som är en kopia av $xs$. \\ \vspace{1em}}
 $n \leftarrow$ antalet element i $xs$ \\
 $ys \leftarrow$ en ny array med plats för $n$ element\\
 $i \leftarrow 0$  \\
 \While{$i < n$}{
  $ys(i) \leftarrow xs(i)$\\
  $i \leftarrow i + 1$\\
 }
 \Return $ys$
\end{algorithm}
\end{Slide}







